\documentclass{scrartcl}
\usepackage{hyperref}
\usepackage{color}
\usepackage{listings}
\usepackage{fontspec}

\definecolor{keyword0}{RGB}{133,153,0}
\definecolor{keyword1}{RGB}{220,50,47}
\definecolor{keyword2}{RGB}{181,137,0}
\definecolor{keyword3}{RGB}{203,75,22}
\definecolor{string}{RGB}{108,113,196}
\definecolor{comment}{RGB}{102,123,131}
\definecolor{background}{RGB}{253,246,227}
\lstdefinelanguage{Perl6} {
    morekeywords={[0]class,role,grammar,method,given,submethod,sub},
    morekeywords={[1]when,rw,required},
    morekeywords={[2]True,False,Bool,Str,Int,Positional,IO},
    morekeywords={[3]has,my,is,unit},
    sensitive=false,
    morecomment=[l]{\#},
    morecomment=[s]{/*}{*/},
    morestring=[b]",
}

\lstset{
    basicstyle=\footnotesize,
    numbers=left,
    keepspaces=true,
    showstringspaces=false,
    showtabs=false,
    keywordstyle={[0]\color{keyword0}\textbf},
    keywordstyle={[1]\color{keyword1}\textbf},
    keywordstyle={[2]\color{keyword2}},
    keywordstyle={[3]\color{keyword3}},
    stringstyle={\color{string}\textit},
    backgroundcolor=\color{background},
    commentstyle=\color{comment}
}
\title{Weave a semi-literate program into Text, Markdown, etc. format}
\begin{document}
\maketitle
\begin{lstlisting}[language=Raku]
    3| 
    4| use File::Temp;
    5| use Semi::Literate;

\end{lstlisting}


\title{Weave a semi-literate program into Text, Markdown, etc. format}

\begin{lstlisting}[language=Raku]
    6| sub MAIN($input-file,
    7|          Bool :l(:$line-numbers)  = True;
    8|          Str :f(:$format) is copy = 'markdown';
    9|          Str :o(:$output-file);
   10|          Bool :v(:$verbose) = True;
   11|     ) {
   12|     my Str  @options;
   13|     my Bool $no-output-file = False;
   14| 
   15|     note "Input Format =>  $format" if $verbose;
   16|     $format .= trim;
   17|     given $format {
   18|         when  /:i ^ markdown | md $ / {
   19|             $format    = 'MarkDown2';
   20|         };
   21|         when  /:i ^ [[plain][\-|_]?]? t[e]?xt $ / {
   22|             $format    = 'Text';
   23|         }
   24|         when  /:i ^ [s]?htm[l]? $/ {
   25|             $format    = 'HTML2';
   26|         } 
   27|         when /:i ^ pdf $ / {
   28|             $format         = 'PDF';
   29|             @options        = "--save-as=$output-file" if $output-file;
   30|             $no-output-file = True;
   31|         }
   32|         when /:i ^ pdf[\-|_]?lite  $ / {
   33|             $format         = 'PDF::Lite';
   34|             @options        = "--save-as=$output-file" if $output-file;
   35|             $no-output-file = True;
   36|         }
   37|         when /:i ^ pod 6? $/ {
   38|             $format    = 'Pod6';
   39|         } 
   40|         when /:i ^ [la]? tex $/ {
   41|             $format    = 'Latex';
   42|         } 
   43|         when /:i man [page]? $/ {
   44|             print "\n\e[33mPod::To::Man may not support pod comment blocks...\e[0m";
   45|             $format    = 'Man';
   46|         } 
   47| 
   48|         default {
   49|             ; 
   50|         } 
   51| 
   52|     } 
   53|     my Str $woven = weave($input-file, :$line-numbers);
   54| 
   55|     my $output-file-handle = $output-file              ??
   56|                                 open(:w, $output-file) !!
   57|                                 $*OUT
   58|                             unless $no-output-file;
   59| 
   60|     if $format eq 'Pod6' {
   61|         $output-file-handle.spurt: $woven;
   62|         return;
   63|     } 
   64| 
   65|     note "Weave Format =>  $format" if $verbose;
   66|     my Str $f = "Pod::To::$format";
   67|     try require ::($f);
   68|     if ::($f) ~~ Failure {
   69|         die "$format is not a supported output format"
   70|     } 
   71| 
   72|     my ($pod-file, $fh) = tempfile(suffix =>  '.rakudoc', :!unlink);
   73|     note "Temp file: $pod-file" if $verbose;
   74| 
   75|     $pod-file.IO.spurt: $woven;
   76| 
   77|     run $*EXECUTABLE,
   78|         "--doc=$format",
   79|         $pod-file,
   80|         @options,
   81|         :out($output-file-handle);
   82| 
   83| } 

\end{lstlisting}
\end{document}
