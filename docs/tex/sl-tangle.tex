\documentclass{scrartcl}
\usepackage{hyperref}
\usepackage{color}
\usepackage{listings}
\usepackage{fontspec}

\definecolor{keyword0}{RGB}{133,153,0}
\definecolor{keyword1}{RGB}{220,50,47}
\definecolor{keyword2}{RGB}{181,137,0}
\definecolor{keyword3}{RGB}{203,75,22}
\definecolor{string}{RGB}{108,113,196}
\definecolor{comment}{RGB}{102,123,131}
\definecolor{background}{RGB}{253,246,227}
\lstdefinelanguage{Perl6} {
    morekeywords={[0]class,role,grammar,method,given,submethod,sub},
    morekeywords={[1]when,rw,required},
    morekeywords={[2]True,False,Bool,Str,Int,Positional,IO},
    morekeywords={[3]has,my,is,unit},
    sensitive=false,
    morecomment=[l]{\#},
    morecomment=[s]{/*}{*/},
    morestring=[b]",
}

\lstset{
    basicstyle=\footnotesize,
    numbers=left,
    keepspaces=true,
    showstringspaces=false,
    showtabs=false,
    keywordstyle={[0]\color{keyword0}\textbf},
    keywordstyle={[1]\color{keyword1}\textbf},
    keywordstyle={[2]\color{keyword2}},
    keywordstyle={[3]\color{keyword3}},
    stringstyle={\color{string}\textit},
    backgroundcolor=\color{background},
    commentstyle=\color{comment}
}
\begin{document}
\begin{lstlisting}[language=Raku]
    3| 
    4| use Semi::Literate;
    5| 
    6| multi MAIN (
    7|     Str $input-file;
    8|     Str :o(:$output-file) = '';
    9| ) {
   10|     my Str $raku-source = tangle $input-file, :!verbose;
   11| 
   12|     my $output-file-handle = $output-file              ??
   13|                                 open(:w, $output-file) !!
   14|                                 $*OUT;
   15| 
   16|     $output-file-handle.spurt: $raku-source;
   17| } 

\end{lstlisting}


\section{NAME}

<application name> - <One line description of application's purpose>

\section{VERSION}

This documentation refers to <application name> version 0.0.1

\section{SYNOPSIS}

\begin{lstlisting}[language=Perl6]
# Brief working invocation example(s) here showing the most common usage(s)

# This section will be as far as many users ever read
# so make it as educational and exemplary as possible.
\end{lstlisting}


\section{REQUIRED ARGUMENTS}

A complete list of every argument that must appear on the command line.
when the application is invoked, explaining what each of them does, any
restrictions on where each one may appear (i.e. flags that must appear
before or after filenames), and how the various arguments and options may
interact (e.g. mutual exclusions, required combinations, etc.)

If all of the application's arguments are optional this section may be
omitted entirely.

\section{OPTIONS}

A complete list of every available option with which the application can be
invoked, explaining what each does, and listing any restrictions, or
interactions.

If the application has no options this section may be omitted entirely.

\section{DESCRIPTION}

A full description of the application and its features. May include
numerous subsections (i.e. =head2, =head3, etc.)

\section{DIAGNOSTICS}

A list of every error and warning message that the application can generate
(even the ones that will "never happen"), with a full explanation of each
problem, one or more likely causes, and any suggested remedies. If the
application generates exit status codes (e.g. under Unix) then list the
exit status associated with each error.

\section{CONFIGURATION AND ENVIRONMENT}

A full explanation of any configuration system(s) used by the application,
including the names and locations of any configuration files, and the
meaning of any environment variables or properties that can be set. These
descriptions must also include details of any configuration language used

\section{DEPENDENCIES}

A list of all the other modules that this module relies upon, including any
restrictions on versions, and an indication whether these required modules
are part of the standard Perl distribution, part of the module's
distribution, or must be installed separately.

\section{INCOMPATIBILITIES}

A list of any modules that this module cannot be used in conjunction with.
This may be due to name conflicts in the interface, or competition for
system or program resources, or due to internal limitations of Perl (for
example, many modules that use source code filters are mutually
incompatible).

\section{BUGS AND LIMITATIONS}

A list of known problems with the module, together with some indication
whether they are likely to be fixed in an upcoming release.

Also a list of restrictions on the features the module does provide: data
types that cannot be handled, performance issues and the circumstances in
which they may arise, practical limitations on the size of data sets,
special cases that are not (yet) handled, etc.

The initial template usually just has:

There are no known bugs in this module. Patches are welcome.

\section{AUTHOR}

Shimon Bollinger (deoac.shimon@gmail.com)

\section{LICENCE AND COPYRIGHT}

© 2023 Shimon Bollinger. All rights reserved.

This module is free software; you can redistribute it and/or modify it
under the same terms as Perl itself. See
\href{http://perldoc.perl.org/perlartistic.html}{perlartistic}.

This program is distributed in the hope that it will be useful, but WITHOUT
ANY WARRANTY; without even the implied warranty of MERCHANTABILITY or
FITNESS FOR A PARTICULAR PURPOSE.\end{document}
